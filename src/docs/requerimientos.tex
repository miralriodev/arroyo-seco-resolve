Universidad Tecnológica de Querétaro.
Integradora
“Evaluación DE”
Grupo: IDSG 11
Alumnos:
Salinas Jiménez María Dolores
Torres Jiménez Ricardo
Miralrío Espinoza Hugo Alberto
Serrano Cruz Hernán
Maestro: Agustín Buenrostro Rico
13/10/2025
Índice
Descripción detallada de la problemática. 2
Objetivos establecidos a corto, mediano y largo plazo. 3
Metas y alcances por entregable. 4
Requerimientos 5
Justificación del tipo de proyecto a desarrollar. 6
Descripción de roles y responsabilidades. 7
Definición de etapas y tiempos. 8
Descripción de los riesgos identificados. 9
Bitácoras. 10
Reportes de seguimiento. 11
Plan de gestión del proyecto. 12
Fuente bibliográfica 13
1
Descripción detallada de la problemática.
Actualmente, no existe una plataforma digital que permita gestionar de manera
eficiente las reservas de alojamiento turístico en la región. Los anfitriones deben
recurrir a métodos manuales o informales para ofrecer sus espacios, mientras que
los visitantes enfrentan dificultades para encontrar, comparar y reservar
hospedajes que se adapten a sus necesidades y presupuesto. Esta falta de
digitalización genera desorganización, falta de visibilidad de la oferta local y poca
confianza entre las partes involucradas.
El proyecto es la respuesta a la ausencia total de una plataforma digital para
reservas de alojamiento y experiencias turísticas en Arroyo Seco, lo cual ha
forzado a anfitriones y visitantes a depender de gestiones manuales e informales.
Esta brecha digital impide el registro estadístico del turismo, limitando la
planificación, la promoción y el desarrollo económico local.
2
Objetivos establecidos a corto, mediano y largo
plazo.
Corto plazo (1 a 6 meses)
 Lograr que todos los anfitriones publiquen sus alojamientos y se registren
50 reservas confirmadas en la plataforma durante los primeros 6 meses de
operación.
 Alcanzar una tasa de satisfacción del 80% entre los usuarios registrados
mediante encuestas digitales aplicadas en los primeros 3 meses posteriores
al lanzamiento.
Mediano plazo (6 meses a 1 año)
 Incrementar en un 50% el número de alojamientos disponibles en la
plataforma antes de cumplir 1 año de operación.
 Incrementar en un 30% los ingresos generados por los anfitriones mediante
reservas gestionadas de forma digital de alojamiento durante los primeros
12 meses de operación.
Largo plazo (1 a 3 años)
 Implementar un módulo de analítica turística que genere reportes
mensuales automatizados sobre flujo de visitantes y ocupación, en un plazo
máximo de 36 meses.
 Incrementar en un 40% el flujo de turistas registrados mediante el uso
continuo de la plataforma digital y la promoción de alojamientos y
experiencias locales, en un periodo máximo de 3 años.
3
Metas y alcances por entregable.
Entregable Meta Alcanzable
Documentación
inicial
Completar la documentación
de planificación, alcance,
objetivos y requerimientos
del proyecto
Documento completo con
alcance, objetivos generales y
específicos, requerimientos
funcionales y no funcionales;
revisión y aprobación por el
equipo
Registro y
acceso de
usuarios
Implementar el flujo de
registro y acceso para
usuarios y anfitriones con
control de roles
Registro de al menos 5 usuarios
de prueba y 5 anfitriones de
prueba; validar 100% del control
de acceso basado en roles
Interfaz de
administración
de
alojamientos
Permitir a los anfitriones
publicar y gestionar
alojamientos
Publicación y gestión de mínimo
10 alojamientos; cambios de
disponibilidad reflejados
correctamente en la base de
datos
Desarrollo
intensivo de la
PWA
Integrar funcionalidades
completas: publicación de
alojamientos, solicitud de
reservas, gestión de
usuarios, interfaz responsive
y controles de seguridad
Registrar 20 reservas de prueba;
interfaz responsive validada en
móviles y ordenadores;
operaciones clave con tiempo de
respuesta < 3 segundos;
seguridad básica implementada
Despliegue y
pruebas finales
Corregir errores detectados,
optimizar rendimiento y
verificar funcionalidad
completa
Todas las funcionalidades
probadas y corregidas; validación
de flujo completo desde registro
hasta reserva; informe de
pruebas documentado
Entrega de
documentación
final
Entregar documentación
completa del proyecto y
manual de usuario
Documentación final revisada y
organizada, incluyendo
diagramas, manual de usuario y
reportes de pruebas
4
Requerimientos
Funcionales:
1. Gestion de usuarios
Descripción: Sistema integral para administrar el registro, autenticación y perfiles
de todos los usuarios de la plataforma (anfitriones y visitantes). Incluye validación
de identidad, verificación documental, personalización de perfiles y gestión de
roles para garantizar seguridad y confianza en las interacciones.
Componentes:
● Registro con validación de identidad
● Verificación de documentos oficiales
● Perfiles personalizables
● Control de roles y permisos
2. Gestión de servicios
Descripción: Plataforma completa para que los anfitriones publiquen, categorizen
y administren sus servicios turísticos. Permite la gestión multimodal de
alojamiento, alimentos, tours, artesanías y paquetes integrados, con control de
disponibilidad e inventario en tiempo real.
Componentes:
● Publicación multimodal de servicios
● Sistema de categorización y etiquetado
● Configuración de disponibilidad
● Gestión de inventario
3. Reservas y pagos
Descripción: Sistema unificado que permite a los visitantes buscar, reservar y
pagar servicios turísticos de forma segura y flexible. Incluye motor de búsqueda
avanzado, carrito para paquetes personalizados y múltiples opciones de pago con
procesamiento seguro.
Componentes:
● Motor de búsqueda con geolocalización
● Sistema de reservas integrado
● Carrito de compras para paquetes
5
● Pasarela de pagos segura
● Pagos fraccionados
● Gestión de comisiones
4. Comunicación
Descripción: Conjunto de herramientas que facilitan la comunicación segura y
efectiva entre anfitriones y visitantes, manteniendo todas las interacciones dentro
de la plataforma para garantizar transparencia y seguridad.
Componentes:
● Sistema de mensajería interno
● Notificaciones push y email
● Sistema de reseñas bidireccional
● Moderación de contenido
5. Personalización
Descripción: Mecanismos inteligentes que adaptan la experiencia de usuario
según preferencias individuales, historial de búsquedas y comportamiento en la
plataforma, mejorando la relevancia de los contenidos y servicios mostrados.
Componentes:
● Algoritmo de recomendaciones personalizadas
● Listas de favoritos
● Perfiles de preferencias
● Contenido adaptativo
6
No Funcionales:
1. Rendimiento
Descripción: Conjunto de características que garantizan que la plataforma
responda de manera rápida y eficiente bajo diferentes cargas de trabajo. Asegura
tiempos de respuesta óptimos, alta disponibilidad y capacidad para manejar
múltiples usuarios concurrentes sin degradación del servicio, proporcionando una
experiencia fluida en cualquier condición de uso.
Componentes:
● Tiempos de respuesta menores a 3 segundos
● Capacidad para 10,000 usuarios concurrentes
● Arquitectura modular escalable
● APIs para integración con terceros
● Monitorización continua del rendimiento
2. Seguridad
Descripción: Marco integral de protección que salvaguarda los datos sensibles de
usuarios y transacciones, previniendo accesos no autorizados y garantizando el
cumplimiento normativo. Implementa múltiples capas de seguridad para construir
confianza y proteger la integridad de la plataforma frente a amenazas cibernéticas.
Componentes:
● Encriptación end-to-end de datos sensibles
● Autenticación de dos factores
● Backup automático diario
● Protección contra inyecciones y fraudes
● Cumplimiento de estándares PCI DSS y GDPR
3. Compatibilidad
Descripción: Capacidad de la plataforma para funcionar consistentemente across
diferentes dispositivos, navegadores y sistemas operativos, garantizando una
experiencia de usuario uniforme y accesible para todos los usuarios
independientemente de su tecnología preferida.
Componentes:
● Diseño responsive para web
● Compatibilidad con iOS y Android
7
● Funcionalidad offline limitada
● Cumplimiento de estándares WCAG 2.1
● Soporte multi-idioma
● Adaptación a diferentes tamaños de pantalla
8
Justificación del tipo de proyecto a desarrollar.
Este proyecto se desarrolla como una plataforma digital integral que responde a la
necesidad moderna de centralizar y digitalizar la oferta turística local. Conecta de
manera eficiente a proveedores de servicios con viajeros, facilitando el acceso a
experiencias auténticas y mejorando la competitividad del sector turístico.
Fundamentos Esenciales:
1. Digitalización del Sector: Moderniza la comercialización de servicios
turísticos tradicionales mediante un canal digital unificado
2. Acceso y Demanda: Responde a la creciente preferencia de viajeros por
plataformas digitales para planificar y reservar experiencias
3. Economicidad: Ofrece a proveedores locales una herramienta accesible
para promocionar sus servicios sin grandes inversiones en marketing
4. Competitividad: Mejora la posición competitiva de la oferta turística local
frente a grandes cadenas y operadores establecidos
5. Experiencia Integral: Soluciona la fragmentación actual al integrar múltiples
servicios turísticos en una sola plataforma
9
Descripción de roles y responsabilidades.
El equipo de 4 personas adoptará los siguientes roles, adaptando la estructura de
Scrum para cubrir las necesidades técnicas y de gestión del proyecto.
Rol Persona
Asignada
Responsabilidades específicas en el proyecto
Líder del
Proyecto
(Product
Owner)
Hugo 1. Visión y Prioridad: Máxima autoridad para definir
y comunicar la visión del proyecto.
2. Gestión de Valor: Crea, mantiene y prioriza el
Product Backlog.
3. Aceptación Final: Única autoridad para aceptar o
rechazar los entregables y firmar la Autorización
Final de Despliegue.
Scrum Master Dolores 1. Coaching y Facilitación: Guía al equipo en el uso
efectivo de Scrum.
2. Gestión de Impedimentos: Elimina obstáculos que
impidan el progreso del equipo.
3. Proceso y Cadencia: Es responsable de asegurar
que el Sprint se complete en la duración de
semanas establecidas.
Arquitecto de
Software
Ricardo 1. Definición Técnica: Es responsable del
Documento de Arquitectura y la selección del Stack
Tecnológico.
2. Calidad y Estándares: Garantiza que los
Principios SOLID y la Cobertura de Pruebas se
cumplan.
3. Seguridad: Diseña e implementa las estrategias
de seguridad informática.
Desarrollador
Principal
Hernán 1. Implementación de Módulos: Lidera el Desarrollo
Web Profesional de los módulos de alta
complejidad.
2. Pruebas y Documentación: Es responsable de
diseñar y ejecutar las Pruebas Unitarias y de
integrar la Documentación Técnica de Desarrollo
Integral.
3. Integración: Lidera la implementación de la
integración de PayPal.
10
Definición de etapas y tiempos.
ETAPA DURACIÓN
ESTIMADA
ACTIVIDADES
PRINCIPALES
ENTREGABLES
Planeación y
análisis
2 semanas Recolección de
requerimientos,
definición del
alcance y análisis
de la problemática.
Documento de
alcance y
requerimientos.
Diseño del sistema 2 semanas Diseño de interfaz
(UI/UX), diagramas
UML, estructura de
base de datos.
Prototipos y
diagramas
técnicos.
Configuración
técnica inicial
2 semanas Configuración de
repositorio,
backend base con
Go/Fiber y
conexión a base de
datos.
API base
conectada a BD.
Módulo de usuarios
y roles
2 semanas Registro, login,
autenticación JWT
y control de roles.
Módulo de
autenticación.
Gestión de
alojamientos
3 semanas CRUD de
alojamientos,
disponibilidad y
galería de
imágenes.
Módulo de
alojamientos.
Gestión de
reservas y pagos
3 semanas Reservas, historial
e integración de
PayPal.
Módulo de reservas
y pagos.
Pruebas e
integración
1 semana Pruebas
funcionales,
integración y
corrección de
errores.
Reporte de
pruebas.
Despliegue y
documentación
1 semana Despliegue final y
documentación
técnica.
MVP desplegado.
11
Descripción de los riesgos identificados.
ID RIESGO
IDENTIFICAD
O
PROBABILIDA
D
IMPACTO ACCIÓN DE
MITIGACIÓN
R-01 Fallas de
comunicación
del equipo
Media Alta Reuniones
semanales y
tablero Scrum.
R-02 Errores en
integración de
PayPal
Media Alta Pruebas en
sandbox y
documentación
.
R-03 Cambios en
requerimientos
Alta Media Gestión en
backlog con
priorización.
R-04 Problemas con
base de datos
Media Alta Backups y
pruebas
constantes.
R-05 Riesgos de
seguridad
Baja Alta Validación,
JWT, cifrado y
HTTPS.
R-06 Fallos en
despliegue
Media Media Pruebas
previas y
rollback plan.
R-07 Retrasos en
desarrollo
Alta Media Sprints y
seguimiento
diario.
R-08 Documentació
n insuficiente
Alta Media Documentació
n continua por
sprint.
12
Bitácoras.
Se implementarán dos tipos de bitácoras para asegurar un registro detallado del
progreso y de los incidentes que surjan durante el ciclo de vida del proyecto.
Bitácora de Sprints: Gestionada por el Scrum Master, esta bitácora registrará el
progreso diario del equipo. Incluirá las tareas completadas, los impedimentos
identificados y resueltos, y las decisiones clave tomadas durante las reuniones
diarias (Daily Scrums). Servirá como un historial detallado de cada sprint para
facilitar las retrospectivas y mejorar la planificación futura.
Bitácora de Incidentes y Riesgos: Este documento centralizará el seguimiento de
todos los riesgos identificados (como los descritos en la sección "Descripción de
los riesgos identificados") y cualquier incidente no planificado que ocurra, como
errores críticos o fallos en el entorno de pruebas. Para cada entrada se registrará:
● ID del incidente/riesgo.
● Descripción detallada.
● Fecha de identificación.
● Impacto y probabilidad evaluados.
● Acciones de mitigación o resolución aplicadas.
● Responsable de la gestión.
● Estado actual (abierto, en progreso, cerrado).
13
Reportes de seguimiento
Para mantener informadas a todas las partes interesadas y asegurar la alineación
con los objetivos del proyecto, se generarán los siguientes reportes de manera
periódica.
● Reporte Semanal de Avance (Sprint Review): Al final de cada semana, el
Scrum Master presentará un informe de estado que incluirá:
○ Porcentaje de avance del sprint actual.
○ Entregables completados y validados por el Product Owner.
○ Desviaciones con respecto a la planificación original.
○ Impedimentos activos que requieren atención.
○ Próximas actividades planificadas.
● Reporte de Cierre de Etapa: Al finalizar cada una de las etapas principales
definidas en el cronograma (ej. "Gestión de alojamientos", "Gestión de
reservas y pagos"), se elaborará un reporte que validará el cumplimiento de
los entregables y metas asociadas a dicha fase.
● Informe Final de Pruebas: Al concluir la etapa de "Pruebas e integración",
se entregará un documento formal que resuma los resultados de las
pruebas funcionales, de rendimiento y seguridad. Detallará los errores
encontrados, las correcciones aplicadas y la validación final del
cumplimiento de todos los requerimientos.
14
Plan de gestión del proyecto.
El proyecto se gestionará bajo un marco de trabajo ágil adaptado de Scrum,
aprovechando su flexibilidad para responder a cambios y su enfoque en la entrega
de valor continuo.
● Gestión del Alcance: El alcance del proyecto está definido por los
requerimientos funcionales y no funcionales. Cualquier cambio o nueva
funcionalidad solicitada será añadida al Product Backlog y priorizada por el
Product Owner (Hugo) para ser considerada en futuros sprints, evitando
así alteraciones no controladas en el sprint actual.
● Gestión del Cronograma: El proyecto se ejecutará en una serie de sprints,
con una duración de dos semanas para las etapas de desarrollo modular. El
Scrum Master (Dolores) será responsable de facilitar las ceremonias de
Scrum (planificación, reuniones diarias, revisión y retrospectiva) para
asegurar que el equipo mantenga el ritmo y cumpla con los plazos
establecidos.
● Gestión de la Comunicación: La comunicación se gestionará a través de
reuniones diarias de seguimiento, reuniones semanales de revisión de
sprint y el uso de un tablero para visualizar el progreso. Esto asegura una
comunicación fluida y transparente entre los miembros del equipo.
● Gestión de Riesgos: Los riesgos identificados serán monitoreados
constantemente por el Scrum Master. La "Bitácora de Incidentes y Riesgos"
será la herramienta central para este fin. En cada reunión de retrospectiva
de sprint, el equipo evaluará la efectividad de las acciones de mitigación y
discutirá nuevos posibles riesgos.
● Gestión de la Calidad: La calidad será asegurada a través de la definición
de estándares de codificación y arquitectura por parte del Arquitecto de
Software (Ricardo), la implementación de pruebas unitarias por el
Desarrollador Principal (Hernán), y una etapa dedicada de pruebas de
integración antes del despliegue final.
15
Fuentes bibliográficas
Clizzz. (2025, febrero 27). Beneficios del servicio de gestión de alojamientos
turísticos. Clizzz.
https://www.clizzz.com/es/beneficios-del-servicio-de-gestion-de-alojamientos-turisti
cos/
Lomelí, L. (2023, mayo 26). Metodología Scrum: roles, procesos y artefactos.
Innevo. https://innevo.com/blog/metodologia-scrum
World Tourism Organization. (2012). Tourism and intangible cultural heritage.
UNWTO. https://www.e-unwto.org/doi/book/10.18111/9789284414796
Fundación Mutua Madrileña, & Fundación ANAR. (2022). La opinión de los
estudiantes (IV Informe de prevención del acoso escolar en centros educativos).
https://www.fundacionmutua.es/actualidad/informe-anar
Organización Mundial de la Salud. (2022, marzo). Preguntas y respuestas sobre
los trastornos del espectro autista (TEA).
https://www.who.int/es/news-room/questions-and-answers/item/autism-spectrum-di
sorders
16