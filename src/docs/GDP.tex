Universidad Tecnológica de Querétaro.
Gestión del proceso de software
1ra evaluación
Grupo: IDSG 11
Alumnos:
Salinas Jiménez María Dolores
Torres Jiménez Ricardo
Miralrío Espinoza Hugo Alberto
Serrano Cruz Hernán
Maestro: Brando Efrén Venegas Olvera
14/10/2025
Caso de Estudio: Plataforma de Gestión de Servicios
Turísticos
El proyecto consiste en desarrollar una plataforma web y móvil que
conecte a turistas con oferentes locales (anfitriones) que brindan servicios
como alojamiento, gastronomía, recorridos, transporte o experiencias
culturales. El objetivo es digitalizar la oferta turística local, mejorar la
visibilidad de los servicios y facilitar la reserva, pago y evaluación de
experiencias en tiempo real.
1. Parámetros de Configuración de las Herramientas Utilizadas
Herramienta Versión /
Configuración
Descripción
Frontend: React + Vite React 18.x, Vite 5.x Creación de la interfaz
de usuario rápida y
modular.
Backend: Node.js
(Express.js)
Node 20.x API REST para manejo
de usuarios, reservas y
pagos.
Base de datos: MySQL v8.x Almacenamiento de
datos de usuarios,
servicios y
transacciones.
Control de versiones:
Git + GitHub
Flujo GitFlow Control de versiones y
colaboración en
equipo.
Pruebas: Postman +
Jest
Configuración base Validación de
endpoints y funciones
críticas.
Despliegue: Render
(backend) + Vercel
(frontend)
CI/CD integrado Entornos de prueba y
producción
automatizados.
Diseño: Figma Prototipado UI/UX Diseño previo de la
interfaz y estructura
visual.
Editor: Visual Studio
Code
ESLint + Prettier Editor principal con
reglas de estilo y
calidad de código.
Configuraciones clave:
- Archivo .env para variables sensibles (puertos, API keys, DB credentials).
- Scripts en package.json para desarrollo, pruebas y despliegue (npm run
dev, npm test, npm run deploy).
- Uso de ESLint para detectar errores de sintaxis y mantener buenas
prácticas.
2. Plan de Pruebas
El plan de pruebas busca asegurar que la plataforma funcione
correctamente en todos sus módulos antes de ser liberada al público.
Objetivos:
- Garantizar la correcta autenticación y registro de usuarios.
- Asegurar que los servicios turísticos se puedan publicar, reservar y
calificar.
- Confirmar la estabilidad y comunicación entre frontend, backend y base
de datos.
Tipos de Pruebas:
1. Unitarias: Validan funciones individuales.
2. Integración: Comprueban interacción entre componentes.
3. Funcionales: Simulan la experiencia del usuario.
4. Rendimiento: Evalúan tiempos de respuesta.
5. Aceptación: Verifican cumplimiento de requerimientos.
Criterios de aceptación:
- Los usuarios pueden registrarse e iniciar sesión sin errores.
- Los anfitriones pueden publicar servicios correctamente.
- Los turistas pueden realizar reservas y dejar comentarios.
- No existen fallos críticos en las rutas principales.
3. Casos de Pruebas
Caso de Prueba Descripción Entrada / Acción Resultado
Esperado
CP-01:
Publicación de
servicio turístico
Verificar que un
anfitrión pueda
registrar un
nuevo servicio.
Llenar
formulario con
nombre,
descripción,
precio y fotos.
El servicio se
guarda y
aparece listado
en la categoría
correspondiente.
CP-02: Reserva
de experiencia
Comprobar que
un turista pueda
reservar una
experiencia
disponible.
Seleccionar
servicio, fecha y
confirmar
reserva.
Se genera una
reserva y se
envía
confirmación al
usuario y
anfitrión.
CP-03:
Calificación y
comentario
Validar que los
usuarios
puedan evaluar
servicios.
Escribir reseña y
asignar
puntuación tras
finalizar la
experiencia.
La calificación se
guarda y
actualiza la
valoración
promedio del
servicio.
CP-04: Filtro y
búsqueda de
alojamientos
Verificar que el
usuario pueda
buscar
hospedajes
según
ubicación,
precio o tipo.
Ingresar
criterios de
búsqueda
(ciudad, rango
de precios, tipo
de alojamiento).
El sistema
muestra una
lista de
alojamientos
que coinciden
con los filtros
aplicados.
CP-05:
Cancelación de
reserva
Validar que un
usuario pueda
cancelar una
reserva activa
dentro del plazo
permitido.
Acceder al
historial de
reservas y
seleccionar
“Cancelar”.
La reserva
cambia su
estado a
“Cancelada” y se
notifica al
anfitrión.
4. Flujo de Trabajo para el Control de Versiones
El flujo de trabajo implementa GitFlow, ideal para equipos distribuidos y
proyectos con múltiples etapas.
1. Rama main: versión estable en producción.
2. Rama develop: contiene las últimas funcionalidades.
3. Ramas feature/nombre-funcion: nuevas características.
4. Ramas test/nombre: validación antes del release.
5. Ramas fix/nombre-error: correcciones en producción.
6. Ramas release/x.x.x: preparan versiones finales.
Buenas prácticas:
- Commits descriptivos (ejemplo: feat: add booking module).
- Pull Requests revisados antes de integrar.
- Etiquetas (tags) para versiones estables.
5. Estrategia de Despliegue
Se adopta un enfoque de despliegue continuo (CI/CD) que permite liberar
versiones estables de forma rápida y segura.
Fases del Despliegue:
1. Entorno de desarrollo: pruebas locales con Docker y MySQL.
2. Entorno de pruebas (staging): despliegue automático en Render y
Vercel.
3. Entorno de producción: versión estable desplegada con GitHub Actions
y respaldos automáticos.
Monitoreo y rollback:
- Logs y métricas (PM2, LogRocket) para detectar fallos.
- Rollback con git checkout a la última versión estable.
6. Artifacts que se implementaran en el proyecto
- Aplicación Móvil (Frontend - React Native): Proporcionar a los usuarios
una interfaz
amigable para consultar opciones de hospedaje y comida en una
ubicación específica.
Incluye módulos para búsqueda, filtrado, visualización de detalles,
calificaciones y
mapas.
- API Backend (Node.js con Express): Gestionar la lógica de negocio, servir
datos a la
aplicación móvil, autenticar usuarios y procesar solicitudes relacionadas
con hospedaje
y comida.
- Base de Datos (MySQL o PostgreSQL): Almacenar información sobre
hospedajes,
restaurantes, usuarios y reservas.
- Panel de Administración (Web - React o Next.js): Permitir a
administradores registrar
nuevos hospedajes, restaurantes, actualizar información y gestionar
usuarios.
- Documentación Técnica: Facilitar la comprensión y mantenimiento del
sistema,
incluyendo manual de usuario, manual de instalación y API docs
(Swagger).
7. Plataformas y herramientas de versionamiento a utilizar
- GitHub: Plataforma principal para alojar el repositorio del proyecto.
- Git: Sistema de control de versiones distribuido.
- GitHub Actions: Automatización de pruebas y despliegue continuo
(CI/CD).
- Docker: Contenerización de los servicios para asegurar portabilidad.
8. Estrategia de despliegue
El proyecto se desplegará bajo un enfoque ágil y de bajo costo, utilizando
servicios Serverless y contenedores ligeros para minimizar los costos
operativos y simplificar el mantenimiento.
Dado que se cuenta con un tiempo limitado de 8 semanas para cubrir la
planeación, desarrollo y despliegue, se priorizará la automatización, la
escalabilidad inmediata y la reducción de tareas de configuración manual.
El entorno principal será Google Cloud Platform (GCP) por su facilidad de
integración con servicios gratuitos y su modelo de facturación flexible. No
obstante, la arquitectura es portable y puede adaptarse a AWS o Azure.
Componentes y Flujo General
1. Frontend (Aplicación Web y Móvil):
- El frontend estará desarrollado con React Native.
- La versión web será desplegada en Firebase Hosting por su
integración gratuita, facilidad de despliegue con un solo comando y
soporte para HTTPS automático.
- La aplicación móvil será compilada y distribuida en Google Play.
2. Backend (API REST Serverless):
- El backend estará implementado con Node.js y Express,
empaquetado como funciones Serverless utilizando Cloud Functions
de GCP (o AWS Lambda si se usa AWS).
- Estas funciones se ejecutan bajo demanda, eliminando la
necesidad de mantener servidores activos, y se escalan
automáticamente según la carga.
- Cada microservicio (usuarios, reservas, pagos, notificaciones, etc.) se
implementará como una función independiente, garantizando
modularidad y aislamiento.
3. Base de Datos:
- Se utilizará PostgreSQL y Firebase dependiendo de la necesidad de
relaciones.
- Ambas opciones ofrecen planes gratuitos, backups automáticos y
alta disponibilidad sin configuración manual.
4. Autenticación y Seguridad:
- Se implementará Firebase Authentication para manejar el inicio de
sesión con correo y redes sociales.
- Las funciones del backend estarán protegidas mediante tokens
JWT y reglas de seguridad a nivel de Firestore o API Gateway.
5. Almacenamiento de Archivos:
- Las imágenes o documentos se alojarán en Firebase Storage o
Cloud Storage, con acceso restringido mediante reglas de seguridad
y URLs firmadas.
6. Balanceo de Carga:
- Aunque las funciones serverless escalan automáticamente, se
utilizará un API Gateway (por ejemplo, Google API Gateway o AWS
API Gateway) para distribuir las solicitudes y definir rutas por
microservicio.
- Esto permite balancear el tráfico y mantener trazabilidad de las
peticiones.
7. CDN (Content Delivery Network):
- Firebase Hosting y Cloud Storage incluyen integración automática
con CDN global, lo que optimiza la entrega de contenido estático
(imágenes, JS, CSS).
8. Panel de Administración (BackOffice):
- El panel de administración se desplegará en Vercel o Netlify,
conectándose directamente a la API Serverless mediante HTTPS
seguro.
9. Monitoreo y Logs:
- Se usará Cloud Logging y Firebase Crashlytics para recopilar
métricas, errores y tiempos de respuesta, lo que permitirá ajustar el
rendimiento sin intervención manual.
10. Repositorio configurado para recibir el código fuente
https://github.com/miralriodev/arroyo-seco-resolve.git
Conclusión
El desarrollo de 'Travel Connect' se apoya en herramientas modernas y
metodologías de control que garantizan calidad, eficiencia y escalabilidad.
El flujo de trabajo basado en Git Flow, junto con un plan de pruebas
estructurado y un despliegue automatizado, asegura entregas continuas y
confiables, alineadas con los objetivos de digitalización y mejora de la
experiencia turística.